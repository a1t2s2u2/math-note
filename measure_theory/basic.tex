\begin{definition}[σ-加法族・可測空間・可測集合]
集合 $X$ に対し,部分集合族 $\mathcal{F}\subset\mathcal{P}(X)$ が次を満たすとき,$\mathcal{F}$ を $X$ 上の σ-加法族という:
\begin{enumerate}
\item $X\in\mathcal{F}$
\item $A\in\mathcal{F}\Rightarrow A^c:=X\setminus A\in\mathcal{F}$(補集合に関して閉じている),
\item $A_n\in\mathcal{F}\ (n=1,2,\dots)\Rightarrow \bigcup_{n=1}^\infty A_n\in\mathcal{F}$(可算和に関して閉じている).
\end{enumerate}
対 $(X,\mathcal{F})$ を可測空間という.集合 $A\subset X$ が $\mathcal{F}$ に属するとき,$A$ を可測集合という.
\end{definition}

\begin{definition}[測度・測度空間]
可測空間 $(X,\mathcal{F})$ 上の写像
\[
\mu:\mathcal{F}\to[0,\infty]
\]
が次の性質を満たすとき,$\mu$ を測度という:
\begin{enumerate}
\item $\mu(\varnothing)=0$,
\item 任意の互いに素な列 $\{A_n\}_{n=1}^\infty\subset\mathcal{F}$ に対し,
\[
\mu\Bigl(\bigcup_{n=1}^\infty A_n\Bigr)=\sum_{n=1}^\infty\mu(A_n).
\]
\end{enumerate}
三つ組 $(X,\mathcal{F},\mu)$ を測度空間という.
\end{definition}

ここで測度は拡張実数値 $[0,\infty]$ を取ることが許される点に注意する.可算加法性から有限加法性・非負性などの基本的性質が従う.特に単調性(定理)が導かれる。

\begin{theorem}[測度の単調性]
測度空間 $(X,\mathcal{F},\mu)$ において,可測集合 $S,T\in\mathcal{F}$ が $S\subseteq T$ を満たすとき,
\[
\mu(S)\le\mu(T)
\]
が成り立つ.
\end{theorem}

\begin{proof}
$S\subseteq T$ と仮定すると,集合の分解
\[
T=S\cup (T\setminus S)
\]
かつ
\[
S\cap (T\setminus S)=\varnothing
\]
が成り立つ.$S,T\setminus S\in\mathcal{F}$ なので可算(ここでは有限)加法性より
\[
\mu(T)=\mu(S)+\mu(T\setminus S).
\]
測度は非負値を取るため $\mu(T\setminus S)\ge0$ であり,従って $\mu(T)\ge\mu(S)$ が得られる.
\end{proof}

\begin{theorem}[測度の単調性]
測度空間 $(X,\mathcal{F},\mu)$ において,
可測集合 $S,T \in \mathcal{F}$ が $S \subseteq T$ を満たすとき,
\[
\mu(S) \le \mu(T)
\]
が成り立つ。
\end{theorem}

\begin{proof}
$S \subseteq T$ と仮定する。
このとき $T$ は
\[
T = S \cup (T \setminus S)
\]
と分解でき,さらに
\[
S \cap (T \setminus S) = \varnothing
\]
が成り立つ。

$S$ および $T \setminus S$ はともに $\mathcal{F}$ に属する(可測である)ので,
測度 $\mu$ の(互いに素な集合に対する)加法性より
\[
\mu(T)
= \mu\bigl(S \cup (T \setminus S)\bigr)
= \mu(S) + \mu(T \setminus S)
\]
が得られる。

測度は非負であるから $\mu(T \setminus S) \ge 0$ であり,
したがって
\[
\mu(T) \ge \mu(S)
\]
が従う。
\end{proof}
